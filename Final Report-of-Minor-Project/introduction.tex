\newpage
\section{Introduction}

\subsection{Introduction to Project}
\textbf{e-Transcript Management System Using Blockchain} is a system which will be used to manage transcripts and store it securely. In the university context, a lot of efforts and costs are put into managing records for all educational transactions occurring since the first-time students submit their papers and register courses
for the first semester until they graduate. In this context, security and data integrity challenges can be identified. The proposed model adds more security via the use of hashing and data readily available with decentralized data storage. 
% The information of blockchain that is provided by the university; will be
% open to all the parties that have interest. Immutability, sensitiveness, and dealing with storing the records in blockchain; altogether will help to get good implementation of student information system. A linked list of blockchains can record all values that include birth certificates, social security cards, student loans, etc. in addition to transactional data such as courses registration and exam marks.

% Now Let us discuss what basically blockchain and transcript Management System is:\\ \\
% \textbf{Blockchain} store data in blocks, that are then linked together via cryptography. As new data comes in, it is entered into a fresh block. Once the block is filled with data it is chained onto the previous block, which makes the data chained together in chronological order. 

% One key difference between a typical database and a blockchain is how the data is structured. A blockchain collects information together in groups, known as blocks, that hold sets of information. Blocks have certain storage capacities and, when filled, are closed and linked to the previously filled block, forming a chain of data known as the blockchain. All new information that follows that freshly added block is compiled into a newly formed block that will then also be added to the chain once filled.

% A database usually structures its data into tables, whereas a blockchain, like its name implies, structures its data into chunks (blocks) that are strung together. This data structure inherently makes an irreversible time line of data when implemented in a decentralized nature. When a block is filled, it is set in stone and becomes a part of this time line. Each block in the chain is given an exact time stamp when it is added to the chain.

% The goal of blockchain is to allow digital information to be recorded and distributed, but not edited. In this way, a blockchain is the foundation for immutable ledgers, or records of transactions that cannot be altered, deleted, or destroyed. This is why blockchains are also known as a distributed ledger technology (DLT).

% \noindent
% \\

\subsection{Project Category}
\begin{itemize}
    % \item Internet based 
    \item Application or System Development
    % \item Research based 
    % \item Industry Automation 
    \item Network or System Administration
    \item Block-chain
\end{itemize}

\noindent
\subsection{Objectives}
This project will help in clarifying issues and benefits to academic institutions and third-parties when processing and viewing transcripts of students in a block chain. \\
The main objectives of this project is:
\begin{enumerate}
    \item {To create a blockchain structure to store transcripts.}
    \item {To design a user interface to access the stored transcripts.}
    \item {To deploy and test the proposed system against existing system.}
\end{enumerate}

\subsection{Problem Formulation}
Based on the analysis of the existing transcript management, it showed problems like:
\begin{itemize}
    \item Centralized record keeping where fault tolerance depends on a single cloud provider.
    \item Record misplacement.
    \item Going back to multiple sources for validating data.
    \item High time consumption.
    \item Non-cost effective system.
\end{itemize}


\subsection{Identification of Need}
Because of the problems inherent in the existing system in the transcript system, the need for computerization becomes imperatives. These are listed below as follows:
\begin{itemize}
    \item Record and reports of students will be easily retrieved with increased data security.
    \item There will be reduction in the amount of resources, which in turn will lower the cost of service, since information will be stored with reduced data Redundancy.
    \item School personnel can attend to students without being over worked.
    \item There will be reduction in time used in retrieval of student files.
    \item Reduction in bulkiness of files and record.
    \item It will make available the storage room that was used for storage of files.
    \item Its facilities easy transaction without delay.
\end{itemize}

\subsection{Existing System}
% A Transcript Management System is an inventory of the courses taken and grades earned of a student throughout a course. The transcript of records provides a standard format for recording all study activities carried out by students. It is an essential tool for academic recognition. Transcript management system is a presentation of student details on a paper in a various academic institution. This is carried out in various ways; such as gathering result form different department in the school, also have a formula for result computation with consideration for carry-overs courses and elective courses for an institution. \\
% \noindent
% Traditional education systems encounter problems such as centralized record keeping where fault tolerance depends on a single cloud provider; not to mention locally hosted databases. A typical transcript request can take up to three weeks.
The existing system of transcript management system involves lots of paper work. The system involves that a student will only get his or her transcript at the end of the program sent to school, organizations that wants to know the students performances in school leading them to know area of strength and weaknesses. Anyone that needs a transcript would have to link his/her way to the university and provide the school authorities with appropriate details before the transcript can be processed during this process lots of error might occur like loss of information.


\subsection{Proposed System}
The implementation of blockchain in the education sector provides a new horizon for set of non-functional requirements including but not limited to: security, immutability, independence from the institution, immutability of official records and certificates. In addition, total trust in the accuracy and infallibility are all gathered in the decentralized ledgers of blockchain. \\
\noindent
By using blockchain technology it is entirely feasible that requesting parties will be provided the requested transcript the same day. When blockchain technology is used in the issue of certificates, there is an opportunity to not only just verify credentials without an intermediary, but also to enrich and add value to the already existent digital certification ecosystem.


\subsection{Unique Features of the System}
This project will help in clarifying issues and benefits to academic institutions and third-parties when processing and viewing transcripts of students in a block chain. 

\begin{enumerate}
    \item \textbf{Secured Data:} One of the objective of Transcript Management system is data security which can be achieved by using Block chain technology as it provides network users with decentralized data storage, where data is stored in blocks and is secured by using some hash value(SHA-256).
    \item \textbf{Time Efficient:} Another objective of Transcript Management system is verification of transcripts in less time. Every person would have their own verifiable learning history and credentials that could be accessed from virtually any device which is an time effective solution.
    \item \textbf{Immutability:} Transcripts are immutable as if one tries to change file, hash of file will be changed and will effect whole blockchain.
    \item \textbf{Visibility:} The data will be visible to only authenticated persons. As certificates uploaded on blockchain, so that certificates will be visible to the person even if any failure comes on one end as this is decentralised system.
\end{enumerate}
