\newpage
\section{Requirement Analysis and System Specification}

\subsection{Feasibility study}
\begin{itemize}
    \item \textbf{Technical Feasibility: } It is technical feasible because user does not need to know anything technical as the interface is very handy. We have created a User Interface (UI) that allows a student to request a transcript from a particular educational institution, typically a university. 
    
    \item \textbf{Economic Feasibility: } Very less time and resources are used for the verification of certificates using blockchain. We are storing hash value(SHA256) as data in the bloackchain, considering very less resources. Moreover, Data is decentralized so no load on single PC.
    
    \item \textbf{Operational Feasibility: } This project can be proven helpful for many requirements like verifiable certificates, easily credit transferable, immutability of records of official papers, certificates and educational organisations. It is operational like any educational organisation can use it. Students will not have to wait for many days to have the needful documents from the college or university. They will get a very good facility from the college because college will be using transcript management system using blockchain.
\end{itemize}

\subsection{Software Requirement Specification Document}
%  which must include the
% following:
% (Data Requirement, Functional Requirement, Performance Requirement
% ,Dependability Requirement, Maintainability requirement, Security requirement, Look and feel requirement)

\begin{itemize}
    \item \textbf{Data Requirement: } The transcript data of students along with their additional details were required to be stored in blocks and to observe the data through repeated careful listening.
    \item \textbf{Functional Requirement: } Blocks check, admin add kr ske
    \item \textbf{Performance Requirement: } 
        \begin{itemize}
        \item \textbf{Performance Requirement: } IPFS Daemon should be on on the server.
        \item \textbf{Software Requirements:} The following specifications are required:
        \begin{itemize}
            \item \textbf{Operating system:} The Certified distribution of Windows, or Linux or MacOS.
            \item \textbf{Front End:} With the aid of Python, a programming language.
            \item \textbf{Back End:} With the aid of Flask and SQL
            \item \textbf{Tools: } IPFS
        \end{itemize} 
        \item \textbf{Hardware Requirements:} The following specifications are required:
        \begin{itemize}
            \item 64-bit  CPU (Intel / AMD architecture) (At least Dual core processor)
            \item 4 GB RAM
            \item At least 5 GB free disk space
        \end{itemize} 
    \end{itemize}
    \item \textbf{Dependability Requirement: } System depends on IPFS server, which is a decentralized storage and data can be easily shared to all peers, to be running, 
    \item \textbf{Maintainability requirement: } This software will be easy to maintain as, using blockchain for transcript Management, data will be decentralized and will be cost efficient.
    \item \textbf{Security requirement: } System security depends on decentralized data management.
    \item \textbf{Look and feel requirement: } E-transcript management system using blockchain is an efficient transcript management system with an easy to use interface where admin can upload transcripts and user/student can access their transcripts with all security. 
\end{itemize}


\subsection{Validation}
The software has been validated and it was found that our pilot implementation is working well.


\subsection{Expected hurdles}
Implementing pilot implementation to the main implementation of the project might become challenging. Hardware requirements may be increased if we implement on the large scale.
\subsection{SDLC model}
\begin{enumerate}
    \item \textbf{Planning and Requirement Analysis: } Requirement analysis is the most important and fundamental stage in SDLC. The information that we got from requirement analysis used to plan the basic project approach and to conduct product feasibility study in the economical, operational and technical areas. Planning for the quality assurance requirements and identification of the risks associated with the project is also done in the planning stage
    \item \textbf{Defining Requirements: } Once the requirement analysis is done the next step is to clearly define and document the product requirements and get them approved from the customer. 
    \item \textbf{Designing the Product Architecture: } The internal design of all the modules of the proposed architecture should be clearly defined with the minutest of the details.
    \item \textbf{Building or Developing the Product: } In this stage of SDLC, the actual development starts and the product is built. The programming code is generated during this stage. Developers must follow the coding guidelines defined by their organization and programming tools like compilers, interpreters, debuggers, etc. are used to generate the code. 
    \item \textbf{Testing the Product: } This stage refers to the testing only stage of the product where product defects are reported, tracked, fixed and retested, until the product reaches the quality standards defined in the SRS.
    \item \textbf{Deployment in the Market and Maintenance: } Once the product is tested and ready to be deployed it is released formally in the appropriate market.
\end{enumerate}

